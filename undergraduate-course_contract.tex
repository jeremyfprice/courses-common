\textit{\textbf{Your first job} is to work at developing an academic relationship with me as your professor} just as you should work to develop a relationship with all of your professors. This job for you extends to \emph{all} of your classes; you will find that putting the effort into building an academic relationship with your professors pays off. Putting effort into building an academic relationship with your professor will make your other two jobs flow much more smoothly. I am (and all of your professors are) here to help you succeed. In order to help me support you, you need to help me know what you need. If you have a question, if you don't understand something, if you are having trouble getting to class, if you need a different explanation of something, \emph{talk to me}. Sometimes this means coming to my Student Drop-In Hours or scheduling an appointment with me outside of class. I work very hard to get to know my students in class, but I get to know them even better (and I can provide more targeted support) when my students talk with me outside of class.

\textit{\textbf{Your second job} is to get to know and learn how to use all of the course materials.} I will provide you with information, models, and scaffolding in the syllabus and other companion documents. These companion documents include \texttt{Reference Sheets}, \texttt{Project Packages}, and \texttt{Model Assignments}. Success in this class involves organizing and understanding these sources of information; we will go over the syllabus in depth during the first class, and I am happy to meet with you to discuss the course and its requirements further.

\textit{\textbf{Your third job} is to consider the long-term returns you will gain from putting effort into the work of this course.} When you are in college, your primary work responsibilities are to your coursework. You are in the Teacher Education Program and the effort you put into your coursework will pay off in terms of becoming a better prepared and more successful teacher. It is hard to find a balance between these long-term payoffs and short term needs (such as paying for rent, clothing, or entertainment and fun activities). You may also be more responsible for your time and attention than you ever have been before, so it is easy to slip into a pattern of missing classes. I am more than happy to help you find the right balance and to connect you with people and resources than can provide further support.

\textit{\textbf{My jobs as your professor}} include designing an engaging, relevant, productive course, facilitating class activities that I believe will be effective in the learning process, assessing your work in a fair, timely manner, and creating a safe, supportive space in which everyone can be who she or he is and freely contribute to the class. I am here to support you as you succeed in this course and in the Teacher Education Program, but I can only do so with your help.

\bigskip

\noindent\small\textit{Thank you to Dr. Terry Murray of the State University of New York at New Paltz for his inspiration (and some of the language) for including a course contract.}\normalsize
