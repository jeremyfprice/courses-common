\section{Participate Effectively to Get More Out of Class Time}
\newthought{We learn best} when we converse with others about ideas and concepts, and participation is an important practice for success in university academic life in general. Ongoing participation is an important and required practice in this course. I understand that this may be out of some students' comfort level, but as you will be educators in the near future, I want to help you develop the skills and confidence to lead a discussion and take intellectual risks\footnote{Another way of thinking about \textit{intellectual risks} is contributing without the fear of being 100\% right. As Ms. Frizzle of the \textit{Magic School Bus} once said, "Take chances, make mistakes, get messy!"}. Teaching is also a highly collaborative career, meaning that you will be working closely with other teachers, school and district administrators, your students and parents just to do your job. Our classroom is a safe environment in which to practice participation skills.

\emph{Quality participation does not mean that you talk the most, or even responding to my questions all the time.}\marginnote{Technology is a wonderful tool, but it is also a way to avoid being present. There is no texting allowed in class, and I will notice when you use the laptops and tablets for reasons other than the task at hand, such as checking Facebook, playing games, or shopping.} Some of the behaviors that show me that you are developing strong participatory and collaborative practices include:
\begin{itemize}
	\itemsep-0.5em
	\item \textbf{Asking questions}
	\item Responding to a \textbf{fellow student}
	\item \textbf{Providing assistance} or helping another student
	\item Making comments drawn from the \textbf{course readings}
	\item Agreeing or disagreeing with something in the text or said in class by the instructor or another student in a way that \textbf{takes the conversation to a new level}
\end{itemize}
It is good to "answer questions" (and sometimes I will ask the class a question to better understand the current level of understanding) and it is often good to draw on your personal experience. But there's more to it than that; participation also involves bringing in what you have learned from the readings and applying what you learn from the in-class discussions to your Learning Performances.
