\section{Fairmont State School of Education Conceptual Framework}
The mission of the Fairmont State University School of Education (FSU SoE) is to prepare reflective and responsive educators who possess the knowledge, skills, and dispositions to help all students learn. The FSU SoE mission is integrated across the curriculum, field experiences, clinical practice, and assessments of candidates. The conceptual framework (CF) provides the structure and guiding principles that are necessary to accomplish this mission. The five West Virginia Professional Teaching Standards (WVPTS) and their respective functions undergird the knowledge, skills, and dispositions that candidates must possess in order to facilitate learning for all students. Diversity and technology are included in the CF representing themes that are integrated throughout the unit's programs. Demonstrated competencies in the standards/functions empower candidates to function as reflective and responsive educators. The CF is based on research about effective teaching and learning best practices that apply to teacher candidates at the initial level as well as accomplished teachers at the advanced level. The CF and the WVPTS also are central guiding elements of the FSU Professional Development School (PDS) Partnership that provides a critical structure and context for teacher education and educator professional development.

\begin{center}
\begin{figure}%
  \centerline{\includegraphics[width=0.5\linewidth]{fsu-cf.png}}
  \caption{Fairmont State University School of Education Conceptual Framework}
  \label{fig:fsu-cf}
\end{figure}
\end{center}

\section{Fairmont State University Policies}

\subsection{Academic Integrity}

Fairmont State values highly the integrity of its student scholars. All students and faculty members are urged to share in the responsibility for removing every situation which might permit or encourage academic dishonesty. Cheating in any form, including plagiarism, must be considered a matter of the gravest concern. Cheating is defined here as: the obtaining of information during an examination; the unauthorized use of books, notes, or other sources of information prior to or during an examination; the removal of faculty examination materials; the alteration of documents or records; or actions identifiable as occurring with the intent to defraud or use under false pretense. Plagiarism is defined here as: the submission of the ideas, words (written or oral), or artistic productions of another, falsely represented as one's original effort or without giving due credit. Students and faculty should examine proper citation forms to avoid inadvertent plagiarism.

\subsection{Disability Services}

Disability services are available to any student, full or part-time, who has a need because of a documented disability. It is the student's responsibility to register for disability services and to provide any necessary documentation to verify a disability or the need for accommodations. Students must provide their professors with a copy of their academic accommodation letter each semester in order to receive accommodations. Faculty, students, and the Office of Disability Services must cooperate to ensure the most effective provision of accommodations for each class.

The Office of Disability Services is located in suite 316 of the Turley Student Services Center 333-3661. For additional information, please visit the Fairmont State University Office of Disability Services webpage at www.fairmontstate.edu/access or call (304) 333-3661.
